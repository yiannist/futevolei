\documentclass[a4paper,11pt]{article}

%% Basics
\usepackage[margin=2cm]{geometry}
\usepackage{graphicx}
\usepackage{hyperref}
\usepackage{xltxtra}
\usepackage{xgreek}
\setmainfont[Mapping=tex-text]{Linux Libertine O}

%% Extras
\usepackage{enumerate}
\usepackage{fancyhdr}
\usepackage{fontawesome}
\usepackage{titling}

\pagestyle{fancy}
\lfoot{\faFacebook \hspace{0.1cm} \texttt{footvolleygreece}}
\rfoot{\faEnvelopeO \hspace{0.1cm} \texttt{mceventss@hotmail.com}}
\renewcommand{\chaptername}{}
\renewcommand{\headrulewidth}{0pt}
\renewcommand{\footrulewidth}{0.4pt}

\pretitle{%
  \begin{center}
  \LARGE
  \includegraphics[
    width=0.15\textwidth,
    keepaspectratio
  ]{../images/footvolley-greece-reduced.pdf}\\[\bigskipamount]
}
\posttitle{\end{center}}
%\setlength{\droptitle}{-1cm}

\title{$1^{o}$ Ευρωπαϊκό Τουρνουά Footvolley Αγρίνιο\\ Κανονισμοί Διοργάνωσης}
\date{}

\begin{document}

\maketitle
\thispagestyle{fancy}

\vspace{-2cm}

Το τουρνουά ακολουθεί τους επίσημους κανόνες της Ευρωπαϊκής Ομοσπονδίας
Ποδοβόλεϊ \cite{EFVL}.

\begin{enumerate}

\item Το τεχνικό meeting θα γίνει το απόγευμα της Παρασκευής 21 Σεπτεμβρίου και
   όλοι οι παίκτες πρέπει να είναι παρόντες.

\item Τα χρηματικά έπαθλα για τις τρεις καλύτερες ομάδες της διοργάνωσης
  ανέρχονται σε 1400 € για την $1^{\eta}$, 800 € για τη $2^{\eta}$ και 600 € για
  την $3^{\eta}$.

\item Οι καλύτερες οχτώ ομάδες (δηλαδή, όσες προκριθούν από τους ομίλους)
  λαμβάνουν επιπλέον χρηματικό bonus 200 €.

\item Είναι υποχρεωτική η χρήση της μπλούζας της διοργάνωσης.

\item Οι αθλητές μπορούν να επιλέξουν τα shorts της διοργάνωσης ή δικά τους
  αρκεί να είναι ίδια.

\item Στα παιχνίδια χρησιμοποιούνται αποκλειστικά οι μπάλες της διοργάνωσης.

\item Το ύψος του φιλέ για τη συγκεκριμένη διοργάνωση (δοκιμαστικά και με την
  έγκριση της EFVL) βρίσκεται στα 2.15 m.

\item Συμμετέχουν δώδεκα (12) ομάδες χωρισμένες σε δύο ομίλους των έξι (6). Οι
  πρώτες 4 ομάδες από κάθε όμιλο προκρίνονται στα προημιτελικά.

\item Για τη κατάταξη των ομάδων στη φάση των ομίλων παίζουν ρόλο τα εξής
  κριτήρια με τη σειρά που αναγράφονται:

  \begin{enumerate}[i)]
  \item Αριθμός νικών
  \item Αριθμός νικών στην/ις αναμέτρηση/σεις μεταξύ των ισοβαθμούντων ομάδων
  \item Συνολικοί πόντοι υπέρ μείον συνολικοί πόντοι κατά (max.)
  \item Συνολικοί πόντοι υπέρ (max.)
  \item Κλήρωση %ή σετ στα 12 (max. στα 15)
  \end{enumerate}

\item Όλα τα παιχνίδια της φάσης των ομίλων και των προημιτελικών αποτελούνται
  από \textbf{ένα} σετ στους 18 πόντους με μέγιστο όριο τους 25 σε περίπτωση που
  δεν υπάρξει διαφορά 2 πόντων. Οι παίκτες αλλάζουν περιοχές καθ' υπόδειξη του
  διαιτητή κάθε 6 πόντους.

\item Τα παιχνίδια των ημιτελικών, της 3ης θέσης και του τελικού είναι στα δύο
  νικηφόρα σετ των 18 πόντων. Αν χρειαστεί tie-break, είναι στους 15 πόντους.
  \textit{Δεν} υπάρχει μέγιστο όριο πόντων. Παρομοίως, οι παίκτες αλλάζουν
  περιοχές κάθε 6 πόντους στα πρώτα δύο σετ και κάθε 5 στο tie-break.

%% \item Στην αρχή του παιχνιδιού αποφασίζεται ποια ομάδα θα έχει το πρώτο service
%%   με ρίψη κέρματος. Το ίδιο συμβαίνει και στην περίπτωση που χρειαστεί
%%   tie-break.

%% \item Οι παίκτες μπορούν να χρησιμοποιήσουν οποιοδήποτε μέρος του σώματος εκτός
%%   από τα χέρια και τα μπράτσα.

%% \item Ο παίκτης που επιτίθεται ή αμύνεται μπορεί να περάσει στην αντίπαλη
%%   περιοχή χωρίς να ακουμπήσει το φιλέ ή να εμποδίσει την αντίπαλη ομάδα.

%% \item Η ζώνη του service ορίζεται από την πίσω γραμμή της περιοχής εώς και 1m
%%   πίσω από αυτή. Αν το βουνό του service καλύπτει έστω και λίγο τη γραμμή,
%%   \textit{ολόκληρο} το βουνό αποτελεί κομμάτι της περιοχής της ομάδας.

%% \item Κατά τη διάρκεια του service, ο συμπαίκτης δεν μπορεί να εμποδίζει τους
%%   παίκτες της αντίπαλης ομάδας να δουν την μπάλα (π.χ. κουνώντας τα χέρια του,
%%   πηδώντας ή κρύβοντας την πορεία της μπάλας).

%% \item Αν μπει μια εξωτερική μπάλα ή κάποιο άλλο αντικείμενο στο γήπεδο κατά τη
%%   διάρκεια του παιχνιδιού, ο διαιτητής οφείλει να διακόψει το παιχνίδι και να
%%   επαναληφθεί το service.

\item Time-out μπορεί να ζητηθεί από την φάση των προημιτελικών και μετά. Το
  time-out διαρκεί 60 δευτερόλεπτα και η κάθε ομάδα έχει ένα time-out ανά σετ.

\item Σε περίπτωση που χρειαστεί διακοπή για ιατρικό λόγο, ο διαιτητής
  αποφασίζει το χρόνο και τη διάρκεια της διακοπής.

\item Αν μία ομάδα δεν είναι έτοιμη να αγωνιστεί εντός 5 λεπτών από την ώρα που
  θα κληθεί στο γήπεδο, χάνει το σετ 18x0 (ή 15x0 αν είναι το tie-break).

%% \item Δεν μπορεί να αντικατασταθεί παίκτης από κάποια ομάδα από τη στιγμή που θα
%%   ξεκινήσουν οι αγώνες.

\item Οι αθλητές οφείλουν να φορούν ρούχα της διοργάνωσης στον αγωνιστικό χώρο
  π.χ. δεν επιτρέπεται η χρήση εμφάνισης από άλλη διοργάνωση ή άλλο χορηγό.
  Επιπλέον, κατά τη διάρκεια της απονομής των επάθλων, όλοι οι αθλητές οφείλουν
  να φορούν την αγωνιστική μπλούζα. Σε διαφορετική περίπτωση παραιτούνται
  \textbf{αυτομάτως} από οποιοδήποτε χρηματικό έπαθλο.

\item Η οργανωτική επιτροπή διατηρεί το δικαίωμα να απορρίψει ή να αποκλείσει
  οποιοδήποτε παίκτη ή ομάδα που δημιουργεί προβλήματα στην ομαλή διεξαγωγή των
  αγώνων. Αυτό σημαίνει ότι αυτομάτως ο παίχτης ή η ομάδα παραιτείται από
  οποιοδήποτε χρηματικό έπαθλο. Τυχούσες ελλείψεις ή αμφισβητούμενες περιπτώσεις
  του παρόντος κανονισμού θα αναλύονται και θα λύνονται από την οργανωτική
  επιτροπή.

\end{enumerate}

%\renewcommand{\section}[2]{}% Hide section name "Αναφορές"
\renewcommand{\refname}{}

\begin{thebibliography}{1}

\bibitem[EFVL]{EFVL}
  European Footvolley League:
  \textit{Official Footvolley Rules}
  \\\texttt{https://footvolleyeurope.com/official-footvolley-rules/}

%% \bibitem[CGF]{CGF}
%%   Circuito Gaúcho de Futevôlei:
%%   \textit{Regulamento}
%%   \\\texttt{https://www.circuitogauchodefutevolei.com/regulamento}
\end{thebibliography}

\end{document}
