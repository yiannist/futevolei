\documentclass[a4paper,11pt]{article}

%% Basics
\usepackage[margin=2cm]{geometry}
\usepackage{graphicx}
\usepackage{hyperref}
\usepackage{xltxtra}
\usepackage{xgreek}
\setmainfont[Mapping=tex-text]{Linux Libertine O}

%% Extras
\usepackage{enumerate}
\usepackage{fancyhdr}
\usepackage{fontawesome}
\usepackage{titling}

\pagestyle{fancy}
\lfoot{\faInstagram/\faFacebook \texttt{Footvolley Athens}}
\rfoot{\faEnvelopeO \texttt{info@futevolei.gr}}
\renewcommand{\chaptername}{}
\renewcommand{\headrulewidth}{0pt}
\renewcommand{\footrulewidth}{0.4pt}

\pretitle{%
  \begin{center}
  \LARGE
  \includegraphics[
    width=0.4\textwidth,
    keepaspectratio
  ]{../images/Footvolley-logo-cropped-reduced.pdf}\\[\bigskipamount]
}
\posttitle{\end{center}}
%\setlength{\droptitle}{-1cm}

\title{Footvolley Athens Premier Tournament \\ Tournament Regulations}
\date{}

\begin{document}

\maketitle
\thispagestyle{fancy}

\vspace{-3cm}

\begin{enumerate}

\item The technical meeting starts at 10:00 and all players should be present.

\item The players must wear the singlet of the tournament.

\item The players can wear whatever shorts or sungas they want.

\item All games of the tournament will be played with balls of Footvolley Athens
  \textbf{exclusively}.

\item If one team is not ready to play within 10 minutes from the time it is
  called to proceed in court, it loses the set 15x0 or 18x0 (depending on the
  phase of the tournament).

%% \item \underline{\hspace{0.4cm}} teams will participate in the tournament
%%   divided in 3 groups.

%% \item Three (3) out of four (4) courts will be used for games and the 4th court
%%   will be shared among the players for warming up.

%% \item The best two teams from each group and the two best third teams proceed to
%%   the quarter-finals.

\item The standings in the group phase are formed based on the following
  criteria in order of priority:

  \begin{enumerate}[i)]
  \item Number of wins
  \item Win(s) in games between two (or more) teams
  \item Points scored for minus points scored against (max.)
  \item Points scored for (max.)
  \item Points scored against (min.)
  \item Draw %ή σετ στα 12 (max. στα 15)
  \end{enumerate}

\item During the group phase, the team that lost the last game should arbitrate
  the next one. In case \textit{that} team has to play the next game, the
  winning team should be the referees of the next game. From the quarter-finals
  and on, the referees will be players that got eliminated. In the end of each
  game, the referees should notify the secretariat of the exact score.

\item Under \textbf{no} circumstances can a player touch the net. This is a
  fault and the referee should register the point to the opponent team.

\item The ball sent to the opponent’s court must go over the net and between the
  two antennas and their imaginary extension. If the ball crosses the net
  outside this area or hits any of the antennas or anything outside from this
  area, it is considered a fault and the opponent team wins the point.

\item If there is any external interference (e.g. a ball from another court
  enters the court) during the game, the play has to be stopped and the rally is
  replayed.

\item A time-out can be requested from the quarter-final phase and on. A
  time-out lasts for 1 minute and each team can request a maximum of one
  time-out per set.

\item All games in the group phase are comprised of \textbf{one} set of 15
  points and at most 19 points in case a minimum lead of two points is not
  achieved. The teams switch courts after every 5 points played.

\item The quarter-finals and the $3^{rd}/4^{th}$-position game are comprised of
  \textbf{one} set of 18 points; the play is continued until a two-point lead is
  achieved. The teams switch courts after every 6 points played.

\item The semi-finals and the final are played until one team wins two sets. One
  set is played until one team wins 18 points with a minimum lead of two points
  (\textit{no} maximum points). In the case of a 1-1 tie, the deciding 3rd set
  is played to 15 points with a minimum lead of 2 points. The teams switch
  courts after every 6 points (set 1 and 2) and 5 points (set 3) played.

\item The ball may touch any part of the body, except arms and hands.

\item In every play, the team can pass the ball by performing one, two or three
  hits.

\item A player is not allowed to hit the ball two \textit{consecutive} times.

\item The first service of a set is executed by the team determined by the coin
  toss. The same happens in the beginning of the tie-break.

\item A player of the serving team must not prevent the opponent, through
  individual screening (e.g. by waving arms, jumping or moving sideways), from
  seeing the server AND the flight path of the ball.

\item The service zone is an 1m wide area behind the end line. If the service
  hill covers the line, it belongs to the playing field (the whole hill,
  including basement).

\item A player may enter into the opponent’s space provided that he doesn't
  touch the net or this does not interfere with the opponent’s play.

\item It is not possible to change a team after the beginning of the tournament.

\item All players that participate in this tournament get 5 points for the
  participation. Additionally, players that classify in the first 8 positions
  gain extra points: $1^{st}$ 10 points, $2^{nd}$ 9 points, $3^{rd}$ 8 points,
  $4^{th}$ 7 points, $5^{th}-8^{th}$ 6 points. This rankings system,
  \textit{``iBeach Footvolley Rankings''}, will be used in future tournaments of
  Footvolley Athens.

\item The organization committee retains the right to reject or disqualify
  whoever player or whichever team creates problems that interfere with the
  smooth conduct of the games. Any deficiencies, doubts or problems of the
  present regulations will be analyzed and resolved by the organization
  committee.

\end{enumerate}


%\renewcommand{\section}[2]{}% Hide section name "Αναφορές"
\renewcommand{\refname}{}

\begin{thebibliography}{1}

\bibitem[EFVL]{EFVL}
  European Footvolley League:
  \textit{Official Footvolley Rules}
  \\\texttt{https://footvolleyeurope.com/official-footvolley-rules/}

\bibitem[CGF]{CGF}
  Circuito Gaúcho de Futevôlei:
  \textit{Regulamento}
  \\\texttt{https://www.circuitogauchodefutevolei.com/regulamento}
\end{thebibliography}

\end{document}
