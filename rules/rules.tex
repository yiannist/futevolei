\documentclass[a4paper,11pt]{article}

%% Basics
\usepackage[margin=2cm]{geometry}
\usepackage{graphicx}
\usepackage{hyperref}
\usepackage{xltxtra}
\usepackage{xgreek}
\setmainfont[Mapping=tex-text]{Linux Libertine O}

%% Extras
\usepackage{enumerate}
\usepackage{fancyhdr}
\usepackage{fontawesome}
\usepackage{titling}

\pagestyle{fancy}
\lfoot{\faInstagram/\faFacebook \texttt{Footvolley Athens}}
\rfoot{\faEnvelopeO \texttt{info@futevolei.gr}}
\renewcommand{\chaptername}{}
\renewcommand{\headrulewidth}{0pt}
\renewcommand{\footrulewidth}{0.4pt}

\pretitle{%
  \begin{center}
  \LARGE
  \includegraphics[
    width=0.4\textwidth,
    keepaspectratio
  ]{../images/Footvolley-logo-cropped-reduced.pdf}\\[\bigskipamount]
}
\posttitle{\end{center}}
%\setlength{\droptitle}{-1cm}

\title{Footvolley Athens Premier Tournament \\ Κανονισμοί Διοργάνωσης}
\date{}

\begin{document}

\maketitle
\thispagestyle{fancy}

\vspace{-3cm}

\begin{enumerate}

\item Το τεχνικό meeting ξεκινάει στις 10:00 και όλοι οι παίκτες πρέπει να είναι
  παρόντες.

\item Είναι υποχρεωτική η χρήση της μπλούζας της διοργάνωσης.

\item Ο αθλητής επιλέγει το shorts ή το μαγιώ που θα φορέσει.

\item Στα παιχνίδια χρησιμοποιούνται \textbf{αποκλειστικά} μπάλες του Footvolley
  Athens.

%% \item Συμμετέχουν \underline{\hspace{0.4cm}} ομάδες χωρισμένες σε 3 ομίλους.

%% \item Χρησιμοποιούνται τα 3 από τα 4 γήπεδα για παιχνίδια και το 4ο γήπεδο το
%%   μοιράζονται οι αθλητές για ζέσταμα.

%% \item Προκρίνονται οι 2 πρώτες ομάδες από κάθε όμιλο και οι δύο καλύτερες
%%   τρίτες.

\item Για τη κατάταξη των ομάδων στη φάση των ομίλων παίζουν ρόλο τα εξής
  κριτήρια με τη σειρά που αναγράφονται:

  \begin{enumerate}[i)]
  \item Αριθμός νικών
  \item Αριθμός νικών στην/ις αναμέτρηση/σεις μεταξύ των ισοβαθμούντων ομάδων
  \item Συνολικοί πόντοι υπέρ μείον συνολικοί πόντοι κατά (max.)
  \item Συνολικοί πόντοι υπέρ (max.)
  \item Κλήρωση %ή σετ στα 12 (max. στα 15)
  \end{enumerate}

\item Όλα τα παιχνίδια στη φάση των ομίλων αποτελούνται από ένα σετ στους 15
  πόντους με μέγιστο όριο τους 19 σε περίπτωση που δεν υπάρξει διαφορά 2 πόντων.
  Οι παίκτες αλλάζουν περιοχές καθ' υπόδειξη του διαιτητή κάθε 5 πόντους.

\item Τα παιχνίδια των προημιτελικών και ο μικρός τελικός αποτελούνται από ένα
  σετ στους 18 πόντους χωρίς μέγιστο όριο πόντων. Οι παίκτες αλλάζουν περιοχές
  καθ' υπόδειξη του διαιτητή κάθε 6 πόντους.

\item Τα παιχνίδια των ημιτελικών και του τελικού είναι στα δύο νικηφόρα σετ των
  18 πόντων. Αν χρειαστεί tie-break, είναι στους 15 πόντους. Δεν υπάρχει μέγιστο
  όριο πόντων. Παρομοίως, οι παίκτες αλλάζουν περιοχές κάθε 6 πόντους στα πρώτα
  δύο σετ και κάθε 5 στο tie-break.

\item Στη φάση των ομίλων, η ομάδα που έχασε το τελευταίο παιχνίδι διαιτητεύει
  το επόμενο παιχνίδι. Αν η ηττημένη ομάδα οφείλει να αγωνιστεί στο επόμενο
  παιχνίδι, τότε διαιτητεύει η νικήτρια ομάδα. Στις επόμενες φάσεις διαιτητεύουν
  ομάδες που έχουν αποκλειστεί. Μετά το τέλος του παιχνιδιού, οι διαιτητές
  ενημερώνουν τη γραμματεία για το ακριβές σκορ.

\item Στην αρχή του παιχνιδιού αποφασίζεται ποια ομάδα θα έχει το πρώτο service
  με ρίψη κέρματος. Το ίδιο συμβαίνει και στην περίπτωση που χρειαστεί
  tie-break.

\item Οι παίκτες μπορούν να χρησιμοποιήσουν οποιοδήποτε μέρος του σώματος εκτός
  από τα χέρια και τα μπράτσα.

\item Η κάθε ομάδα δικαιούται να περάσει την μπάλα χρησιμοποιώντας 1 εώς 3
  επαφές.

\item Απαγορεύεται κάποιος παίκτης να κάνει δύο \textit{συνεχόμενες} επαφές.

\item Σε \textbf{καμία} περίπτωση δεν επιτρέπεται να ακουμπήσει κάποιος από τους
  παίκτες το φιλέ. Αυτό αποτελεί παράβαση και ο διαιτητής κατοχυρώνει τον πόντο
  στην αντίπαλη ομάδα.

\item Ο παίκτης που επιτίθεται ή αμύνεται μπορεί να περάσει στην αντίπαλη
  περιοχή χωρίς να ακουμπήσει το φιλέ ή να εμποδίσει την αντίπαλη ομάδα.

\item Η μπάλα πρέπει να περνάει στην απέναντι περιοχή πάνω από το φιλέ και
  ανάμεσα των δύο κεραιών ή των νοητών προεκτάσεων αυτών. Αν η μπάλα περάσει
  εκτός αυτών ή ακουμπήσει κάποια εξ αυτών ή την περιοχή του φιλέ εκτός αυτών,
  καταλογίζεται παράβαση και η αντίπαλη ομάδα κερδίζει τον πόντο.

\item Η ζώνη του service ορίζεται από την πίσω γραμμή της περιοχής εώς και 1m
  πίσω από αυτή. Αν το βουνό του service καλύπτει έστω και λίγο τη γραμμή,
  \textit{ολόκληρο} το βουνό αποτελεί κομμάτι της περιοχής της ομάδας.

\item Κατά τη διάρκεια του service, ο συμπαίκτης δεν μπορεί να εμποδίζει τους
  παίκτες της αντίπαλης ομάδας να δουν την μπάλα (π.χ. κουνώντας τα χέρια του,
  πηδώντας ή κρύβοντας την πορεία της μπάλας).

\item Αν μπει μια εξωτερική μπάλα στο γήπεδο κατά τη διάρκεια του παιχνιδιού, ο
  διαιτητής οφείλει να διακόψει το παιχνίδι και να επαναληφθεί το service.

\item Time-out μπορεί να ζητηθεί από την φάση των προημιτελικών και μετά. Το
  time-out διαρκεί 1' και η κάθε ομάδα έχει ένα time-out ανά σετ.

\item Αν μία ομάδα δεν είναι έτοιμη να αγωνιστεί εντός 10 λεπτών από την ώρα που
  θα κληθεί στο γήπεδο, χάνει το σετ 15x0 ή 18x0 (αναλόγως από τη φάση της
  διοργάνωσης).

\item Δεν μπορεί να αντικατασταθεί παίκτης από κάποια ομάδα από τη στιγμή που θα
  ξεκινήσουν οι αγώνες.

\item Όλοι οι αθλητές που θα συμμετέχουν στο τουρνουά παίρνουν 5 πόντους για τη
  συμμετοχή. Οι αθλητές που θα τερματίσουν μεταξύ των θέσεων $1^{\eta}$ και
  $8^{\eta}$ παίρνουν επιπλέον: $1^{\eta}$ 10 πόντους, $2^{\eta}$ 9 πόντους,
  $3^{\eta}$ 8 πόντους, $4^{\eta}$ 7 πόντους, $5^{\eta}-8^{\eta}$ 6 πόντους.
  Αυτό το σύστημα βαθμολογιών, \textit{``iBeach Footvolley Rankings''}, θα
  χρησιμοποιηθεί και στα επόμενα τουρνουά του Footvolley Athens.

\item Η οργανωτική επιτροπή διατηρεί το δικαίωμα να απορρίψει ή να αποκλείσει
  οποιοδήποτε παίκτη ή ομάδα που δημιουργεί προβλήματα στην ομαλή διεξαγωγή των
  αγώνων. Τυχούσες ελλείψεις ή αμφισβητούμενες περιπτώσεις του παρόντος
  κανονισμού θα αναλύονται και θα λύνονται από την οργανωτική επιτροπή.

\end{enumerate}

%\renewcommand{\section}[2]{}% Hide section name "Αναφορές"
\renewcommand{\refname}{}

\begin{thebibliography}{1}

\bibitem[EFVL]{EFVL}
  European Footvolley League:
  \textit{Official Footvolley Rules}
  \\\texttt{https://footvolleyeurope.com/official-footvolley-rules/}

\bibitem[CGF]{CGF}
  Circuito Gaúcho de Futevôlei:
  \textit{Regulamento}
  \\\texttt{https://www.circuitogauchodefutevolei.com/regulamento}
\end{thebibliography}

\end{document}
